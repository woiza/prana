\chapter{Zitate und Referenzen}
\label{cha:ZitateReferenzen}
% 
Dieses Kapitel beschäftigt sich mit Verweisen -- zum einen mit internen, aber zum anderen und im Besonderen
mit dem Verweis auf verwendete Quellen durch Zitieren. Eine \gls{faq} in diesem Zusammenhang ist die nach 
dem \Fachbegriff{Zitationsstil}. 
%
\todo{Welche Zitationsstile sind möglich?} 
%

Daneben geht es auch um besondere Formen interner Referenzen, etwa für besondere Verzeichnisse. Ferner
gibt es einige weitere Hinweise, etwa zum \gls{doi} oder auf Software zur Verwaltung der Quellen.

Außerdem werden in der
Quelldatei \Datei{Inhalt/ZitateReferenzen.tex} auch einige der in der Datei \Datei{Formales/Befehle.tex} 
definierten eigenen Befehle verwendet, um auch dies zu exemplifizieren.
%
\section{Zitate}
\label{sec:ZR-Zitate}
%
Zunächst einmal sind beim Zitieren bestimmte \emph{Prinzipien} zu beachten. Das sicher wichtigste
dieser Prinzipien lautet:
%
\begin{quote}
\textsl{Hat man etwas aus Quellen
übernommen (Text, Ideen, Abbildungen, Code, etc.), so muss man die Quelle auch angeben!}
\end{quote}
%
Anderenfalls handelt es sich um ein \gls{plagiat}, was insbesondere bei Vorsatz zur Bewertung der
Abschlussarbeit mit \textbf{ungenügend} führt!

Hinsichtlich Zitaten stellt sich außerdem immer die Frage nach dem \emph{Stil}. Oftmals wird mit Fußnoten
gearbeitet, was aber nicht den Vorgaben entspricht. Eigentlich handelt es sich lediglich um eine einzige:
%
\begin{quote}
\textsl{Es ist der \Fachbegriff{Harvard-Stil}, auch \Fachbegriff{Autor-Jahr-Zitierweise} genannt, mit
Kurzbelegen zu verwenden! \parencite[s.][]{Wikipedia2010}}
\end{quote}
%
Wie dies umzusetzen ist, kann man bereits in den Kapiteln \ref{cha:Einleitung} und \ref{cha:Forschungsmethoden}
in Verbindung mit dem Literaturverzeichnis sehen. Dort erkennt man eine weitere Anforderung, nämlich:
%
\begin{quote}
\textsl{Reine Onlinequellen sind von wissenschaftlichen Quellen wie Büchern und Artikeln in 
(wissenschaftlichen) Publikationen zu trennen!}
\end{quote}
%

Weitere nützliche Hinweise zum korrekten Zitieren findet man auf der Website der Uni Ulm 
bei \textcite{Hoelting2018}. Zusätzlich ist auch zu beachten, dass man keine \Fachbegriff{Blindquellen}
ins Literaturverzeichnis aufnimmt, \dh keine Quellen, die nicht auch wirklich in der Arbeit zitiert
werden. Um dies sicher zu stellen ist eine weitere Anforderung:
%
\begin{quote}
\textsl{Im Literaturverzeichnis sind Rückverweise zu verwenden!}
\end{quote}
% 
Dies bedeutet, dass im Literaturverzeichnis bei den Quellen anzugeben ist, auf welcher Seite der Arbeit
sie verwendet wurden. Mehr zum Thema Rückverweise findet sich in Abschnitt \ref{subsec:ZR-BR-Rueckverweise}.
%
\todo{Das obige ToDo kann weg} 
%
\section{Besondere Referenzen}
\label{sec:ZR-BesondereReferenzen}
%
Es werden bei Verwendung dieser \LaTeX{}-Vorlage auch Referenzen für das Abkürzungs- und Symbolverzeichnis
sowie für das Glossar erzeugt (Informationen hierzu findet man in kompakter Form bei \textcite{Partosch2015}).
Ferner wird aus dem Literaturverzeichnis heraus auf die Seiten zurückverwiesen, auf denen ein Zitat vorkommt.
Diese Arten von besonderen Referenzen sollen hier kurz diskutiert werden.

Dabei werden Abkürzungen, Symbole und Glossar mit Hilfe des Pakets \Paket{glossaries} und den zugehörigen
Befehlen wie etwa \Code{\bs{}gls\{\}} erzeugt, während für
die Rückverweise eine Kombination von Optionen in den Paketen \Paket{biblatex} und \Paket{hyperref} 
verantwortlich sind.
%
\subsection{Abkürzungen}
\label{subsec:ZR-BR-Abkuerzungen}
%
Auch für Abkürzungen ist eine wichtige Regel zu beachten:
%
\begin{quote}
\textsl{Bei der ersten Verwendung einer Abkürzung / eines Akronyms im Text ist
der Begriff zusätzlich auszuschreiben bzw.~zu erklären!}
\end{quote}
%
Abkürzungen werden bei dieser Vorlage, wie Symbole und Glossareinträge auch, in einer speziellen Datei,
nämlich \Datei{Inhalt/Glossar.tex}, gesammelt.
Ein Beispiel ist das Akronym \gls{degeval}. Man beachte, dass bei der ersten Verwendung weiter oben
automatisch der ausgeschriebene Begriff verwendet wurde, während später, wie \zB hier, lediglich
das Akronym kommt,
so wie vorgeschrieben. In jedem Fall wird im PDF auf das Abkürzungsverzeichnis referenziert. Ein
Abkürzungsverzeichnis wird erwartet!
%
\subsection{Symbole}
\label{subsec:ZR-BR-Symbole}
%
Die folgenden Symbole werden exemplarisch definiert und dann ins Symbolverzeichnis übernommen:
zunächst das \gls{kappa} und dann das \gls{datenbasis}. 
%
\subsection{Glossar}
\label{subsec:ZR-BR-Glossar}
%
Das Glossar ist ein besonderes Verzeichnis, in dem man wichtige Begriffe bei Bedarf näher erläutern kann. Wie
das Symbolverzeichnis ist es optional, während ein Abkürzungsverzeichnis eigentlich immer nötig ist. 
%
\subsection{Rückverweise}
\label{subsec:ZR-BR-Rueckverweise}
%
Eine interessante Option für das Literaturverzeichnis sind Rückverweise auf die Seite, auf der die Verweise
erfolgt sind. Dies wird, wie in Abschnitt \ref{sec:ZR-Zitate} erwähnt, auch explizit verlangt, um die
Angabe von Blindquellen auszuschließen.
Für die korrekte Umsetzung mit \LaTeX{} sind sowohl im Paket \Paket{biblatex} die Option \Code{backref} und im
Paket \Paket{hyperref} verschiedene Optionen zu setzen (siehe Datei \Datei{Formales/Packages.tex}).
Darüberhinaus muss man noch in der Hauptdatei \Datei{Abschlussarbeit.tex} mit dem
\Code{\bs{}pagenumbering}-Befehl geschickt arbeiten (siehe dort). Tut man dies nicht, so werden zwar die
Seitenzahlen korrekt im Literaturverzeichnis angegeben, aber die Verweise im PDF sind inkorrekt.
%

Abschließend soll noch erwähnt werden, dass offensichtlich auch interne Verweise auf Abschnitte oder
bestimmte Positionen möglich sind. Dafür ist die Vergabe einer Markierung mittels \Code{\bs{}label\{name\}}
notwendig und die spätere Referenzierung mittels \Code{\bs{}ref\{name\}} oder \Code{\bs{}pageref\{name\}} 
(bzw.~\Code{\bs{}autoref\{name\}} in Verbindung mit dem \Paket{hyperref}-Paket). 
So kann man etwa den Unterschied beim Zitieren zwischen den Befehlen \Code{\bs{}textcite} und 
\Code{\bs{}parencite} im \autoref{citedemo} sehen.
%
\section{Weitere Hinweise}
\label{sec:ZR-WeitereHinweise}
%
%
\subsection{Trennung nach Quellenart}
\label{subsec:ZR-WH-Trennung}
%
Um die Anforderung nach Trennung von Online-Quellen und Büchern / (wissenschaftlichen) Publikationen umzusetzen
wird in dieser Vorlage einfach das Schlüsselwort \Code{internet} für Internetquellen verwendet und dieses dann
beim Erzeugen des Literaturverzeichnisses als Diskriminator nutzt. In der Hauptdatei
\Datei{Abschlussarbeit.tex} sieht das dann in den Zeilen 138 -- 144 wie in \listing{QuellenTrennung} aus:\\
%
\todo[size=\footnotesize]{Zei\-len\-num\-mern prüfen!!}
%
% Als Hintergrund eine Farbe mit Namen (dvipsnames), und davon nur 30%
%
\lstinputlisting[firstline=138, lastline=144, language={[LaTeX]TeX}, %
 caption={Trennung der Quellenarten}, backgroundcolor=\color{SkyBlue!30}, %
 emph={notkeyword,keyword}, emphstyle=\color{YellowOrange}, %
 label=lst:QuellenTrennung,
]{Abschlussarbeit.tex}
%
\paragraph{Umgang mit URLs} 
\label{para:urls}%
Diese sollten bei Büchern und Publikationen \emph{nicht} mit angegeben werden, während
sie bei Online-Quellen natürlich essenziell sind. Daher kann man hier nicht allgemein die Option setzen, um URLs
komplett aus dem Literaturverzeichnis zu verbannen. Allerdings sollte die URL-Information auch nicht ganz unter
den Tisch fallen, zumindest in der Literaturdatei sollte sie vorhanden sein. Daher ist diese in Quellen, die keine
Online-Quellen sind, im \emph{Kommentarfeld} zu hinterlegen.

Gleiches gilt für den Fall, dass eine URL einer Internetquelle im Literaturverzeichnis als deutlich zu lang
empfunden wird oder gar von \LaTeX{} nicht richtig umgebrochen werden kann. In diesem Fall benötigt man allerdings
einen Ersatz in Form eines \Fachbegriff{Shortlinks}, den man beispielsweise mit \textsf{Bitly} 
(\url{https://bitly.com/}) oder \textsf{Google} (\url{https://goo.gl/}) erzeugen kann.
%
\subsection{Verwendung von DOI}
\label{subsec:ZR-WH-DOI}
%
Neuere Bücher und (wissenschaftliche) Publikationen verwenden \glslink{doi}{DOI} als eindeutigen Identifikator
der Quelle. Daher sollte dieser, falls vorhanden, auch in der \emph{Literatur\underline{datei}}, aber
\textbf{nicht} im \emph{Literaturverzeichnis} auftauchen.

Es ist jedoch \emph{Vorsicht geboten}, denn diese DOI stimmen nicht immer (gerade bei älteren Publikationen) und sollten somit auf jeden Fall überprüft werden, denn fehlerhafte DOI in der Literaturdatei führen zur
Abwertung.
%
\subsection{Software}
\label{subsec:ZR-WH-Software}
%
Neben einer \gls{ide} für das Erstellen der \LaTeX{}-Dateien (wie etwa \Software{TeXmaker} (siehe Abschnitt
\ref{subsec:texmaker}) ist auch eine Software für die Verwaltung Ihrer Quellen zu verwenden. Hierbei 
ist eine klare Empfehlung \Software{JabRef}, ein plattformunabhängiges Java-Programm mit umfangreicher
Funktionalität (siehe Abschnitt \ref{subsec:jabref}).

Man beachte, dass \Software{JabRef} die Möglichkeit bietet, direkt aus dem Programm heraus eine
beträchtliche Anzahl an Internetportalen für wissenschaftliche Publikationen zu recherchieren
und anschließend die gefundenen Quellen
direkt zu übernehmen (sowie diese an die \LaTeX{}-\gls{ide} zu \emph{pushen}).

Eine kompakte Einführung mit Beispielen für Internetrecherche und Betrachtung verschiedener 
Literaturverwaltungssoftware gibt \textcite{Partosch2011}. Hier wird allerdings noch Bib\TeX{} verwendet.
Den Umgang mit \Paket{biblatex} mit \Paket{biber}-Backend behandelt \textcite{Pospiech2011} oder
(neuer) \textcite{Frank2017}. Dort findet man auch noch viele weitere Kursteile mit Übungen.
%

\underline{\emph{Bemerkung:}} Die beiden oben genannten Tutorials wurden ebenfalls mit \LaTeX{}
erstellt, nämlich mit dem Paket \Paket{latex-beamer}.
%

In jedem Fall gilt:
%
\begin{quote}
\textsl{Es ist ein Quellenverzeichnis im Bib(La)\TeX{}-Format in elektronischer Form mit abzugeben!}
\end{quote}
%

In den vorangegangenen beiden Kapiteln wurden allgemeingültige Informationen für Abschlussarbeiten
hinsichtlich der einzusetzenden
Forschungsmethoden sowie des Zitierens und anderer Arten von Verweisen gegeben, wobei in diesem Zusammenhang
natürlich bereits zahlreiche \LaTeX{}-Elemente exemplarisch verwendet wurden. Im nächsten Kapitel wird nun auf
spezielle Elemente nochmals näher eingegangen.
%
\todo[color=green!30]{übergeleitet!}
%
