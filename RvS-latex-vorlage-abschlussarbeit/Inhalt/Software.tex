\section{Empfehlenswerte Software}
\label{sec:Software}
%
Eine Arbeit mit \LaTeXe{} wird eher entwickelt als geschrieben. Daher benötigt man dafür neben einer 
\Fachbegriff{Distribution} so etwas wie eine \gls{ide}. Und auch die Literaturverwaltung gestaltet 
sich softwaregestützt deutlich einfacher als das Pflegen einer Textdatei mit einem Editor (obwohl 
dies theoretisch auch möglich wäre, was zumindest gut zu wissen ist). Für diese drei Aspekte
werden in diesem Abschnitt Empfehlungen ausgesprochen

%
\subsection{\LaTeX{}-Distribution}
\label{subsec:distribution}
%
Als allererstes stellt sich die Frage nach der zu verwendenden \LaTeX{}-Distribution. Unter Windows wird normalerweise
\Software{MiKTeX} verwendet, unter Linux hingegen normalerweise die \Software{TeXlive}-Umgebung. Diese ist zumeist in den 
Linux-Distributionen über die Paketverwaltung installierbar. Dabei kann man in beiden Fällen mit einer Minimalversion
beginnen und später etwa fehlende Pakete einfach nachinstallieren.

Mehr Informationen finden sich auf den zugehörigen Websites \url{http://miktex.org/} 
bzw.~ \url{https://www.tug.org/texlive/}.
%
\subsection{TeXMaker}
\label{subsec:texmaker}
%
Dann stellt sich die Frage nach einer \LaTeX{}-\gls{ide}. Hier empfiehlt sich eine 
%
% Trennung des Wortes muss LaTeX beigebracht werden ...
%
platt\-form\-über\-grei\-fen\-de
%
Lösung wie \Software{Texmaker}. Diese bietet neben dem selbstverständlichen \Fachbegriff{Syntax-Highlighting}
auch \Fachbegriff{Auto-Vervollständigung} bei der Eingabe der üblichen \LaTeX{}-Befehle. Die Software ist
sehr gut anpassbar und es lassen sich aus ihr heraus auch direkt weitere Programme aufrufen, die für die
korrekte Übersetzung der Quelldateien benötigt werden (insbesondere \Software{biber} (bzw.~\Software{BibTeX})
und \Software{makeglossaries} oder auch \Software{makeindex}).

Die zugehörige Website ist \url{http://www.xm1math.net/texmaker/}, wo man Weiteres erfährt und die Software
auch herunterladen kann. Unter Linux ist die Wahrscheinlichkeit hoch, dass sie auch direkt in der
Paketverwaltung verfügbar ist.
%
\subsection{Jabref}
\label{subsec:jabref}
%
Mit \Software{Jabref} lässt sich die Literaturverwaltung gut organisieren. Aus dem Programm heraus können
direkt verschiedene Online-Bibliothekskataloge durchsucht und mit Hilfe der Ergebnisse korrekte Einträge
generiert werden. Auch die händische Eingabe von Quellen wird gut unterstützt, da das Programm alle möglichen
Arten von Einträgen vorschlägt und man anschließend nur noch den richtigen auswählen und die relevanten Felder
ausfüllen muss. Auch diese Software ist als Java-Programm plattformunabhängig.

Auf der zugehörigen Webseite \url{http://www.jabref.org/} findet man nähere Informationen und die bereits in
Abschnitt \vref{subsec:ZR-WH-Software} genannten Tutorials helfen beim Einstieg.

Natürlich gibt es für alle diese Programme noch Alternativen, die aber bei Bedarf selbst recherchiert werden
müssen.

Der Hauptteil des Dokuments ist damit abgeschlossen. Es folgt noch das abschließende Fazit nebst kritischer
Bewertung. Untypischer Weise sind hier auch noch die Anhänge \ref{anh:Anh-Anforderungen} und \ref{anh:Anh-Abgabe}
wichtig, da dort die wesentlichen Anforderungen an eine Abschlussarbeit und deren Abgabe zusammengefasst aufgeführt
werden.


